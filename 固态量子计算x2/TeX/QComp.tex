% !Mode:: "TeX:UTF-8"
% !TEX program  = xelatex
\documentclass[a4paper]{article}
\usepackage{amsmath}
\usepackage{amssymb}
\usepackage{ctex}
\usepackage{braket}
%\usepackage[european]{circuitikz}
\usepackage{multirow}
\usepackage{float}
\usepackage{colortbl}
\usepackage{graphicx}
\usepackage{geometry}
\geometry{left=2.5cm,right=2.5cm,bottom=2.5cm,top=2.5cm}
\usepackage{physics}
\usepackage{mhchem}
\usepackage{siunitx}
\usepackage{bm}


\title{近代物理实验报告:单电子固态量子计算}
\author{\quad 学号\quad 匡亚明学院}
\date{2019年2月29日}
\begin{document}
\maketitle
\bibliographystyle{unsrt}

\tableofcontents
\newpage
%--------main-body------------

\section{引言}


%\section{实验目的}
\subsection{量子计算背景}
过去的几十年中,经典计算机经历了快速的发展时期。第一台通用电子计算机ENIAC
占地约170平方米,如今的掌上电脑已经可以放进口袋。体积的巨大变化,主要归功
于集成电路工业的飞速发展。英特尔公司创始人之一戈登·摩尔曾提出著名的摩尔定
律,用以总结和预期集成电路的发展,即集成电路上可容纳的晶体管数目,约每隔18
个月便会翻一倍,其性能也会翻倍。然而随着电路集成度越来越高,摩尔定律也遇到
了新的挑战。因为按照摩尔定律描述的发展趋势,集成电路的工艺己进入纳米尺度。
在芯片上如此高密度的集成元器件,热耗散问题是一个巨大的挑战。更严重的是,随
着集成电路的工艺进入纳米尺度,量子效应会逐渐显现并占据支配地位。当描述元器
件工作的物理规律由经典物理转变为量子力学,试图按照原来的方式保持集成电路的
发展趋势就非常困难了。

既然在微观尺度下,量子力学效应占主导,那有没有可能利用量子力学效应来构
造计算机呢?费曼最先在1982年指出,采用经典计算机不可能以有效方式来模拟量子
系统的演化。我们知道,经典计算机与量子系统遵从不同的物理规律,用于描述量子
态演化所需要的经典信息量,远远大于用来以同样精度描述相应的经典系统所需的经
典信息量。费曼提出用量子计算则可以精确而方便地实现这种模拟。1985 年,David
Deutsch深入研究了量子计算机是否比经典计算机更有效率的问题。他首次在理论上
描述出了量子计算机的简单模型——量子图灵机模型,研究了它的一般性质,预言了
它的潜在能力。但当时的人们还不知道有什么具体的可求解问题,量子计算能比经典
计算更有优越性。1994年,美国数学家Peter Shor从原理上指出,量子计算机可以用
比经典计算机快得多的速度来求解大数的质因子分解问题。由于大数质因子分解问题
是现代通信与信息安全的基石,Shor的开创性工作引起了巨大的关注,其可期待的辉
煌应用潜力有力地刺激了量子计算机和量子密码等领域的研究发展,成为量子信息科
学发展的重要里程碑之一。1996 年Grover发现了另一种很有用的量子算法,即所谓
的量子搜索算法,它适用于解决如下问题: 从$ N $个未分类的客体中寻找出某个特定的
客体。经典算法只能是一个接一个地搜寻,直到找到所要的客体为止,这种算法平均
地讲要寻找$ N/2 $次,成功几率为$ 1/2 $,而采用Grover的量子算法则只需要
$ \sqrt{N} $次。

随着一系列量子算法的提出,量子计算对某些重要问题相对于己知的经典计算方
式的计算能力的展现出巨大的优势。量子计算不仅吸引着众多的科研人员,其应用前
景也吸引了谷歌、微软、IBM 等国际知名公司参与这一领域的竞争。近年来,各研究团队更是试图实现“量子霸权”(Quantum supremacy),即通过量子计算实现对经典计算能力的极限的突破

\section{实验原理}
\subsection{量子计算基本概念}
经典计算机需要信息的载体,逻辑操作,状态读出等一系列基本元素。量子计算机也
类似,首先我们需要量子信息的载体,即量子比特。然后需要具备对量子比特进行初
始化,操控和读出的能力。我们利用一系列的逻辑操作,构成量子算法,来实现特定
的计算目的。
\subsubsection{量子比特}

\subsubsection{量子逻辑门}
\subsubsection{量子测量}
\subsubsection{量子算法}


\subsection{量子计算的实验实现}



\section{实验内容及结果}
\subsection{连续波实验}
\subsection{拉比振荡实验}
\subsection{$ T_2 $实验}
\subsection{D-J算法实验}
\subsection{设计实验}




%\section{实验数据}


\section{思考题}
\subsection*{请利用布洛赫球表示以下量子态:}

\subsection*{如果实验中施加的微波频率$ f $与共振频率$ f_0 $有偏差,即$ f = f_0 + \delta f $,拉比振荡的频率会如何变化?}

\subsection*{拉比振荡频率与微波功率的关系是什么?}

\subsection*{参照$ n=1 $的特殊情况,即图1.5所示的量子线路图,画出一般情况的D-J算法量子线路图,并解释算法原理}


%\nocite{jiaocai}
\bibliography{ref}
\end{document}